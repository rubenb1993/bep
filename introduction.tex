\section{Introduction}
In the year 2010, the most common cause of death was a heart disease, and the fourth most common cause of death was a cerebrovascular disease. Combined, they were responsible for over 29\% of all deaths in the United States of America (cite national vital statistics reports). With such a high mortality rate, understanding the cause of these diseases is very important. 

Most of the deaths relating to heart and cerebrovascular disease have to do with atherosclerosis. Atherosclerosis is the build up of plaque around the artery wall, which hardens and narrows the artery. This inhibits the flow of blood, which is vital for all organs to operate.

With multiple papers showing the correlation between atherosclerosis and wall shear stress on the artery wall (cite atherosclerosis papers), it could be vital in understanding the flow of blood to make more accurate predictions of the development of these diseases. 

When blood flow in arteries is better understood, predictions can be made of injected drugs distributions and models may even be extended to include the effect of magnetic fields on these distributions. 






This paper will discuss the effects of the viscoelastic behaviour of the non-Newtonian blood on the artery walls and to quantitatively compare different models for these viscous effects. 